\documentclass[11pt,british,a4paper]{report}
%\pdfobjcompresslevel=0
%\usepackage{pythontex}
\usepackage[usenames,dvipsnames]{xcolor}
\usepackage[includeheadfoot,margin=0.8 in]{geometry}
\usepackage{siunitx,physics,cancel,upgreek,varioref,listings,booktabs,pdfpages,ifthen,polynom,todonotes}
%\usepackage{minted}
\usepackage[backend=biber]{biblatex}
\DefineBibliographyStrings{english}{%
      bibliography = {References},
}
\addbibresource{sources.bib}
\usepackage{mathtools,upgreek,bigints}
\usepackage{babel}
\usepackage{graphicx}
\graphicspath{{./}{./e/}}
\usepackage{float}
\usepackage{amsmath}
\usepackage{amssymb,epstopdf}
\usepackage{enumitem}
\usepackage[T1]{fontenc}
%\usepackage{fouriernc}
% \usepackage[T1]{fontenc}
\usepackage{mathpazo}
% \usepackage{inconsolata}
%\usepackage{eulervm}
%\usepackage{cmbright}
%\usepackage{fontspec}
%\usepackage{unicode-math}
%\setmainfont{Tex Gyre Pagella}
%\setmathfont{Tex Gyre Pagella Math}
%\setmonofont{Tex Gyre Cursor}
%\renewcommand*\ttdefault{txtt}
\usepackage[scaled]{beramono}
\usepackage{fancyhdr}
\usepackage[utf8]{inputenc}
\usepackage{textcomp}
\usepackage{lastpage}
\usepackage{microtype}
\usepackage[font=normalsize]{subcaption}
\usepackage{luacode}
\usepackage[linktoc=all, bookmarks=true, pdfauthor={Anders Johansson},pdftitle={FYS-MEK-project}]{hyperref}
\usepackage{tikz,pgfplots,pgfplotstable}
\usepgfplotslibrary{colorbrewer}
\usepgfplotslibrary{external}
\tikzset{external/system call={lualatex \tikzexternalcheckshellescape -halt-on-error -interaction=batchmode -jobname "\image" "\texsource"}}
\tikzexternalize[prefix=tmp/, mode=list and make]
\pgfplotsset{cycle list/Dark2}
\pgfplotsset{compat=1.8}
\renewcommand{\CancelColor}{\color{red}}
\let\oldexp=\exp
\renewcommand{\exp}[1]{\mathrm{e}^{#1}}
\renewcommand{\Re}[1]{\mathfrak{Re}\ifthenelse{\equal{#1}{}}{}{\left(#1\right)}}
\renewcommand{\Im}[1]{\mathfrak{Im}\ifthenelse{\equal{#1}{}}{}{\left(#1\right)}}
\renewcommand{\i}{\mathrm{i}}
\newcommand{\tittel}[1]{\title{#1 \vspace{-7ex}}\author{}\date{}\maketitle\thispagestyle{fancy}\pagestyle{fancy}\setcounter{page}{1}}

% \newcommand{\deloppg}[2][]{\subsection*{#2) #1}\addcontentsline{toc}{subsection}{#2)}\refstepcounter{subsection}\label{#2}}
% \newcommand{\oppg}[1]{\section*{Oppgave #1}\addcontentsline{toc}{section}{Oppgave #1}\refstepcounter{section}\label{oppg#1}}

\labelformat{section}{#1}
\labelformat{subsection}{exercise~#1}
\labelformat{subsubsection}{paragraph~#1}
\labelformat{equation}{equation~(#1)}
\labelformat{figure}{figure~#1}
\labelformat{table}{table~#1}

\renewcommand{\footrulewidth}{\headrulewidth}

%\setcounter{secnumdepth}{4}
\renewcommand{\thesection}{Part \arabic{section}}
\renewcommand{\thesubsection}{\alph{subsection})}
\renewcommand{\thesubsubsection}{\arabic{section}\alph{subsection}\roman{subsubsection})}
\setlength{\parindent}{0cm}
\setlength{\parskip}{1em}

\definecolor{bluekeywords}{rgb}{0.13,0.13,1}
\definecolor{greencomments}{rgb}{0,0.5,0}
\definecolor{redstrings}{rgb}{0.9,0,0}
\lstset{rangeprefix=!/,
    rangesuffix=/!,
    includerangemarker=false}
\renewcommand{\lstlistingname}{Kodesnutt}
\lstset{showstringspaces=false,
    basicstyle=\small\ttfamily,
    keywordstyle=\color{bluekeywords},
    commentstyle=\color{greencomments},
    numberstyle=\color{bluekeywords},
    stringstyle=\color{redstrings},
    breaklines=true,
    %texcl=true,
    language=Fortran
}
\colorlet{DarkGrey}{white!20!black}
\newcommand{\eqtag}[1]{\refstepcounter{equation}\tag{\theequation}\label{#1}}
\hypersetup{hidelinks=True}

\sisetup{detect-all}
\sisetup{exponent-product = \cdot, output-product = \cdot,per-mode=symbol}
% \sisetup{output-decimal-marker={,}}
\sisetup{round-mode = off, round-precision=3}
\sisetup{number-unit-product = \ }

\allowdisplaybreaks[4]
\fancyhf{}

\rhead{MD-Project}
\rfoot{Page~\thepage{} of~\pageref{LastPage}}
\lhead{FYS-MEK1110}

%\definecolor{gronn}{rgb}{0.29, 0.33, 0.13}
\definecolor{gronn}{rgb}{0, 0.5, 0}

\newcommand{\husk}[2]{\tikz[baseline,remember picture,inner sep=0pt,outer sep=0pt]{\node[anchor=base] (#1) {\(#2\)};}}
\newcommand{\artanh}[1]{\operatorname{artanh}{\qty(#1)}}
\newcommand{\matrise}[1]{\begin{pmatrix}#1\end{pmatrix}}


\pgfplotstableset{1000 sep={\,},
                      assign column name/.style={/pgfplots/table/column name={\multicolumn{1}{c}{#1}}},
                      every head row/.style={before row=\toprule,after row=\midrule},
                      every last row/.style={after row=\bottomrule},
                      columns/n/.style={column name={\(n^*\)},column type={r}},
                      columns/N/.style={column name={\(N\)},sci},
                      columns/logN/.style={column name={\(\log(N)\)}},
                      columns/logn/.style={column name={\(\log(n^*)\)}}
                      }

\newread\infile

%start
\begin{document}
%\maketitle

\begin{titlepage}
%\includegraphics[width=\textwidth]{fysisk.pdf}
\vspace*{\fill}
\begin{center}
\textsf{
    \Huge \textbf{Molecular Dynamics Project}\\\vspace{0.5cm}
    \Large \textbf{FYS-MEK1110 - Mechanics}\\
    \vspace{8cm}
    Tommy Myrvik and Anders Johansson\\
    \today\\
}
\vspace{1.5cm}
\includegraphics{uio.pdf}\\
\vspace*{\fill}
\end{center}
\end{titlepage}
\null
\pagestyle{empty}
\newpage

\pagestyle{fancy}
\setcounter{page}{1}

\section{Introduction}
In this project you will learn the basics of a simulation technique called molecular dynamics (MD). Molecular dynamics is a method actively used in research here at the Department of Physics, yet its basic principle can be understood and implemented with the background of a first-year physics student.

Molecular dynamics is based on the assumption that even atoms move according to the laws of Newton, given the correct model for interactions. The goal of this project is to model an argon gas, where the atoms interact according to the famous Lennard-Jones potential,
\begin{equation}
    U(r) = 4\varepsilon\qty(\qty(\frac{\sigma}{r})^{12} - \qty(\frac{\sigma}{r})^6), \label{eq:lj}
\end{equation}
where \(r\) is the distance between two atoms, \(r=\norm{\vec{r}_i-\vec{r}_j}\). \(\sigma\) and \(\varepsilon\) are a parameters which determine which chemical compound is modelled. This potential is a good approximation for noble gases.

\subsection{Understanding the potential}
\begin{enumerate}[label=\roman*.]
    \item Plot the potential with \(\varepsilon=1\) and \(\sigma=1\).
    \item The behaviour of \(U(r)\) is vastly different for \(r \ll \sigma\) and \(r \gg \sigma\). Which term in the potential,~\vref{eq:lj}, dominates in each case and what is the effect?
    \item Find and characterise the equilibrium points of the potential.
    \item Describe qualitatively the motion of two atoms which start at rest separated by a distance of \(\num{1.5}\sigma\).
\end{enumerate}

\subsection{Forces and equations of motion}
\begin{enumerate}[label=\roman*.]
    \item Find the force on atom \(i\) at position \(\vec{r}_i\) from atom \(j\) at position \(\vec{r}_j\).
    \item Show that the equation of motion for atom \(i\) is
    \begin{equation}
        \dv[2]{\vec{r}_i}{t} = \frac{24\varepsilon}{m} \sum_{j \neq i} \qty(2\qty(\frac{\sigma}{\norm{\vec{r}_j-\vec{r}_i}})^{12}-\qty(\frac{\sigma}{\norm{\vec{r}_j-\vec{r}_i}})^6)\frac{\vec{r}_j-\vec{r}_i}{\norm{\vec{r}_j-\vec{r}_i}^2}.
    \end{equation}
\end{enumerate}

\subsection{Units}
As you may remember from MAT-INF1100, numerical accuracy is reduced when computing with values which are many orders of magnitude apart. This is often an issue in physics, and molecular dynamics is no exception. For example, the mass of argon i smaller than \(10^{-25}\ \si{\kg}\), while typical length scales are on the order of nanometres, \(10^{-9}\ \si{\m}\).

The remedy is to change units so that most quantities are close to \(1\). From \vref{eq:lj} it is clear that \(\sigma\) and \(\varepsilon\) are the typical scale for length and energy.

\begin{enumerate}[label=\roman*.]
    \item Introduce the scaled coordinates \(\vec{r}_i\,'=\vec{r}_i/\sigma\) and show that the equation of motion can be rewritten in terms of these coordinates as
    \begin{equation}
        \dv[2]{\vec{r}_i\,'}{{t'}} = 24 \sum_{j \neq i} \qty(2\norm{\vec{r}_j\,'-\vec{r}_i\,'}^{-12}-\norm{\vec{r}_j\,'-\vec{r}_i\,'}^{-6})\frac{\vec{r}_j\,'-\vec{r}_i\,'}{\norm{\vec{r}_j\,'-\vec{r}_i\,'}^2}.
    \end{equation}
\item What is the characteristic time scale \(t'\), and what is its value for argon, which has \(\sigma=\SI{3.405}{\angstrom}\) (\(\SI{1}{\angstrom}=\SI{1e-10}{\m}\)) and \(\varepsilon=\SI{1.0318e-2}{\eV}\) (\(\SI{1}{\eV}=\SI{1.602e-19}{\J}\))?
\end{enumerate}













\section*{\underline{Part 1: 2-atom model}}

\textbf{Bare legger en liten mal her på oppgavene jeg har programmert så langt. Veldig overfladisk oppgavetekst, kan og bør endres underveis.}

\subsection*{a)}

Plot the LJ-potential curve. What does the different terms in the potential do?

\subsection*{b)}

Find the force corresponding to the potential from a), and plot the result. For what distance $r$ is this force 0? Is this force conservative? \textit{What's different with this force
compared to say Newton's Law of gravitation?}

\subsection*{c)}

Say we place one atom at $\vec{r_1} = [0, 0, 0]$ and the other at $\vec{r_2} = [1.5, 0, 0]$, with no initial velocities. Describe qualitaively what the motions of the atoms will look like. Support your arguments with the LJ potential curve.

If we place the second atom at $\vec{r_2} = [0.95, 0, 0]$ instead, what do you expect then? Explain by again using the potential curve from task a).

\subsection*{d)}

Develop a code to simulate a system of two atoms with the LJ-forces being the only ones acting. Use the Euler-Cromer integration method (?). Simulate the system with the initial conditions
(both cases) from task c), and plot the distance between the atoms $r$ as a function of time. What do you see?

\subsection*{e) (optional)}

Download a visualization tool (i.e. Ovito), and write a function that writes the positions of the atoms at every time step to a xyz.-file. Load this file into \textit{Ovito} and describe
what you see. Does this fit well with the assumptions you made in task c) ?

\subsection*{f)}

Implement different integration methods so that you have the Euler, Euler-Cromer and Velocity-Verlet methods available for your simulations. Run simulations with the same initial conditions
as in the first case in task c) ($\vec{r_1} = [0, 0, 0]$, $\vec{r_2} = [1.5, 0, 0]$) for all three methods, and plot the mechanical energy for all three with $\Delta t = 0.01$. Compare
the results for all three methods. How does these methods perform in terms of energy conservation?

\subsection{g) (I tvil om denne skal med)}

Run simulations for all three methods implemented in the previous task, and find the an approximation to the largest time step ($\Delta t$) required to keep the integration from exploding.
Whats the difference between these methods, and why does some of them perform better than others?



\section*{\underline{Part 2: N-atom model}}


\section*{\underline{Part 3: Let's do some science!}}

\nocite{*}
\printbibliography{}
\addcontentsline{toc}{chapter}{\bibname}
\end{document}
